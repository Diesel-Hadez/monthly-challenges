\documentclass[twocolumn, a4paper,12pt]{article}
\usepackage{fontspec}
\usepackage[paper=a4paper]{geometry}
\usepackage{amsmath}
\usepackage{amssymb}
\usepackage{hyperref}
\usepackage{algorithm2e}
\usepackage{tabularx}
\usepackage{microtype}
\newfontfamily\cjkfont{Noto Sans CJK SC}

% Shuts microtype up
% Because it doesn't know chinese
\makeatletter
\def\MT@warn@unknown{}
\makeatother

% Because algorithm2e thinks it doesn't support twoclumn
% https://tex.stackexchange.com/a/82272
\makeatletter
\newcommand{\removelatexerror}{\let\@latex@error\@gobble}
\makeatother


\pagestyle{empty}

%opening
\title{A Short Introduction to Brute-Forcing RSA}
\author{\cjkfont 赖纳文}
\date{November 16 2020}

% Modular Congruence and Equivalence uses the same symbol
% For better clarity, create an alias for equiv to modcongr
\let\modcong\equiv

\begin{document}
\maketitle

%\section*{Acknowledgements}
% \begin{itemize}
%  \item \LaTeX 
%  \item \LaTeX package {\emph algorithm2e}
%  \item \LaTeX package {\emph ifoddpage}
%  \item \LaTeX package {\emph relsize}  
% Get rid of everything below in octopi
%  \item \LaTeX package {\emph glossaries}  
%  \item \LaTeX package {\emph mfirstuc}  
%  \item \LaTeX package {\emph xfor}  
%  \item \LaTeX package {\emph datatool-base} 
%  \item \LaTeX package {\emph datatool}  
%  \item \LaTeX package {\emph substr}  
% \end{itemize}


\begin{abstract}
This paper, as the title suggests, introduces a brief overview on how to brute-force RSA cryptography.
This could either mean 
given the encryption key and modulus, the decryption key would be retrieved, 
or given the decryption key and modulus, the encryption key would be retrieved. 

A brief explanation of RSA and how to use it is also
covered, although proof of its correctness is out-of-scope of this paper.
\footnote{For the curious, See section VI of the original RSA paper\cite{Rivest1978} which
makes use {\em Fermat's Little Theorem}, or alternatively here: \href{https://www.di-mgt.com.au/rsa\_theory.html}{https://www.di-mgt.com.au/rsa\_theory.html}}
\end{abstract}

\section{Introduction}
RSA is an assymmetric, or public-key cryptosystem,
meaning that there are separate keys used for encryption and decryption.
Details of the RSA cryptosystem were first publicly described in 1977 and a formal paper was published in 1978. \cite{Rivest1978} The basic premise is that the cryptosystem relies on 
the difficulty of factoring two prime numbers. 

That is to say, where p and q are prime,
and given:-
\[ n = p \cdot q \]

It would be ``difficult'' (i.e: could not be done in polynomial time
\footnote {For now, I advise to just think of it [polynomial time] as ``
the time scaling depending on the value at a reasonable rate''. If you are familiar with Big O notation, polynomial time is in Big O notation $O(n^{c})$, don't worry too much if you don't understand.}
) to find 
$p$ and $q$ given simply just $n$ and $e$ or $d$.
\footnote{Technically speaking, we don't have mathematical proof it isn't possible
(on classical computers at least, on Quantum computers there is an algorithm called Shor's
Algorithm for this), it's an open unsolved problem in computer science} 
$n$ could be factored 
in only one way using primes (excluding rearrangements), 
as proven by the {\em Fundamental Theorem of Arithmetic}.
\footnote{This was proven by Euclid thousands of years ago and you
can find a proof online.}


This process is called integer 
factorisation \footnote{ Or rather, {\emph factorization} for our American counterparts}.

\section{Process}
There are several processes involved, namely, the {\em encryption} process, {\em decryption}
process, and the {\em key generation} process. We will be working backwards, starting with
encryption/decryption first, so we can understand {\em how} RSA is used, then move on to
how to generate RSA keys. First off, some key terms:-

\subsubsection{\texorpdfstring{$n$}{n} - \emph{modulus}} 
The product of prime numbers $p$ and $q$. 
This is given out along with the public key $d$.
Sometimes when references are made to the ``public key'', it is implied that this modulus
is included. See RFC8017 \cite{rfc8017} Section A.1.1 for more information. Not to be confused with the `Absolute Value' function of no relation
also called Modulus.

\subsubsection{\texorpdfstring{$e$}{e} - \emph{encryption key}}
Also known as the public key. It is used to encrypt messages.
Normally the number $65537$ is used\footnote{Notice how $65537$ is actually $2^{16}+1$,
which is actually called a Fermat prime. It's used because it's faster to do certain
calculations (one of the modular exponentiation algorithms) 
because this number in binary form is $10000.....1_{2}$ and only has two 1s.}, although
technically any number which is between 1 and the totient function (see next the totient function
sections below) and is coprime\footnote{
Also known as relatively prime or mutually prime. Basically an integer where the greatest
common divisor of itself and the number it is coprime with is 1.
} with the totient function can be used.

\subsubsection{\texorpdfstring{$d$}{d} - \emph{decryption key}}
Also known as the private key. It is used to decrypt messages.

\subsubsection{\texorpdfstring{$C$}{C} - \emph{ciphertext}}
The message in its encrypted form.

\subsubsection{\texorpdfstring{$P$}{P} - \emph{plaintext}}
The message before encryption, or equivalantly: the encrypted message after being decrypted.

\subsubsection{\texorpdfstring{$\phi$}{phi} - \emph{Euler's totient function}}
This was the totient function used in the original RSA paper.\cite{Rivest1978} 
The result of $\phi (n) $ is the number of positive integers less than $n$ that are coprime with $n$ itself. 

Euler's product formula states
\[ \phi(n) = n \prod\limits_{p | n}{(1-\frac{1}{p})}\]

But there is a short
way to do this in our case, given $n=p \cdot q$ where $p$ and $q$ are primes, $\phi (n)$ can
be derived from $p$ and $q$ like so\footnote{See: \href{http://www.math.unl.edu/~tmarley1/math189/notes/Nov20notes.pdf} {http://www.math.unl.edu/~tmarley1/math189/notes/Nov20notes.pdf} for proof} 
:-
\[ \phi = (p - 1) \cdot (q - 1) \]

\subsubsection{\texorpdfstring{$\lambda$}{lambda} - \emph{Carmichael's totient function}}
This is an improved `reduced' totient function that is used in
place of Euler's totient function sometimes. It's used in the PKCS\#1 RFC8017\cite{rfc8017}, which is a specification for an RSA implementation. It gives
a different and smaller number than Euler's totient function, while
still functioning the same in context of RSA. $\lambda (n)$ can
be derived from primes $p$ and $q$ like so:-
\[ \lambda = lcm( (p - 1) , (q - 1) ) \]

where $lcm$ is the lowest common multiple.\footnote{There exists
a formula for calculating the lcm using the gcd (which you can, of course,
find using the Euclidean algorithm), $lcm(a,b) = 
\frac{|a \cdot b|}{gcd(a,b)}$ . Proof not provided.}

Now, we can move on to the actual processes
\subsection{Encryption}
The ciphertext $C$ can be derived from the plaintext, encryption key, and modulus like so:-
\[ C \modcong P^e \mod n \]

\subsection{Decryption}
The plaintext $P$ can be derived from the ciphertext, decryption key, and modulus like so:-
\[ P \modcong C^d \mod n  \]
You may notice that since it is $\mod n$, the resulting plaintext can must be less than or
equal to the modulus ($P \leqslant n$), and yes that is indeed true. For that reason
we can only encrypt messages which in length are less than or equal to the modulus. In
practice, very large numbers for $p$ and $q$ are used and this isn't a problem.
\footnote{Furthermore, since RSA is `slow', usually a separate faster
symmetric encryption system is used, and RSA is simply used to encrypt the key for
that cryptosystem, which need not be so long.}

\subsection{Key Generation}
First, compute the modulus $n$ as the product of two arbritatrily chosen primes.
\footnote{If the purpose was for stronger encryption, they won't be so arbritarily chosen.
Generally, bigger numbers are better as it will take longer to break. The original 
paper \cite{Rivest1978} recommends around 100 digits for $p$ and $q$ to end up with 
a roughly 200 digit modulus $n$. It also recommends for the two numbers to 
differ in length by a few digits to protect against ``sophisticated factoring algorithms''
which is not covered in this paper. For the curious, this is through Fermat Factorisation,
which you are likely familiar with from Secondary School, even if you didn't know the name
of it. It is difference of two squares. $a^{2} - b^{2} = (a+b) \cdot (a-b)$}

\[ n = p \cdot q \]

Then we need to find $d$ and $e$ such that:-

\[ d \cdot e \modcong 1 \mod \phi (n) \]

or alternatively

\[ d \cdot e \modcong 1 \mod \lambda (n) \]

That is to say, $d$ and $e$ are modular multiplicative inverses of each other 
$\mod \phi (n)$ or $\lambda (n)$ .

You then choose either the decryption key $d$ or the encryption key $e$ by being choosing 
an integer which is coprime to a totient function (either one. Euler's totient function was
used in the original RSA paper, Carmichael's totient function is the `improved' version), 
see above.


Note that some sources may say something like `determine $d = e^{-1} \modcong \phi (n) $ ' which
is an abuse of notation since you cannot solve that since $e^{-1}$ is not an integer value. 

Instead, solve it using the Extended Euclidean Algorithm like a 
normal linear congruence. For example, for an linear congruence 
$3x \modcong 1 \mod 5$, change it into the form $3x - 5k = 1$ where 
$k$ is an integer, and find the greatest common divisor using 
the Euclidean Algorithm of 3 and 5, then use the Extended 
Euclidean Algorithm to find an integer value
for $x$. Do recall that there are multiple integer solutions for $x$ and $k$ (
Refer to lecture notes and recorded lectures and/or tutorials), usually we will use
a positive value for $x$ since that is easier to work with.

\subsection{Fast Modular Exponentiation {\em (extra)}}
Not strictly neccessary for small numbers, but for a large number
you may be able to optimise modular exponentiaion (used in encryption/decryption) 
by splitting the $\mod$ for each variable. For example,
an example using $m$ is 5, the modulus is 22, and $e$ is 13.\footnote{
This example was taken directly from one of the MAT1830 Lectures.}
\begin{equation*}
\begin{split}
 C & \modcong m^e \mod n \\
 C & \modcong 5^{13} \mod 22 \\
 C & \modcong {(5^2)}^3 (5) \mod 22 \\
 C & \modcong {(25 \mod 22)}^3 (5) \mod 22 \\
 C & \modcong {(3)}^6 (5) \mod 22 \\
 C & \modcong (729 \mod 22) * {5 \mod 22} \\
 C & \modcong 15 \mod 22
\end{split}
\end{equation*}

Thus we don't have to compute the large number $5^{13}$ and can 
simply work with smaller numbers like $3^6$ and $5$, which is especially
useful if you are doing 64-bit calculations and resulting numbers
of $m^e$ are larger than that. \footnote{Python, as always, already has
a function that does modular exponentiation, and in any case, numbers in python
aren't restricted to any number of bits. But I thought this was an interesting
thing to discuss, and it would come in handy if you were working in a lower-level
language.}

Details of implementation in code is left as an exercise to the reader.\footnote{
Some key terms which may be useful are `modular exponenetiation' or `modpow', with the `pow' standing for `power'.
}

\section{Brute-Forcing to find p and q}
Finally on to the most interesting part.
From subsection \hyperlink{subsection.2.3}{\textbf{2.3}} we can
see an equation involving $d$, $e$ and $n$. Great! We have $e$ and
$n$, it should be trivial to get $d$ right, by way of Euler's Extended Algorithm? 


Not exactly. We need to calculate $\phi (n)$ or $\lambda (n)$ to use as the modulus. 
Previously we used a short way by using either Euler's 
totient function or Carmichael's totient
functions; but that only works if we know the values of $p$ and $q$!

For now, let's assume the key generation is using Euler's totient function. If you don't
recall, Euler's totient function returns the number of integers less than $n$ that are coprime
with $n$ itself. And so we start with this basic algorithm to find $\phi (n)$

\begingroup
\removelatexerror% Nullify \@latex@error
\begin{algorithm*}[H]
\SetAlgoLined
\KwIn{modulus $n$}
\KwOut{Euler's totient function (number of integers < $n$ that is coprime to $n$)}

set counter to 0\;
\For{numbers in range 1 through $n$ inclusive}{
    \If{gcd(current\_number, $n$) is equal to 1}{
        add 1 to counter\;
    }
}
return counter\;
\caption{Euler's Totient Brute Force}
\end{algorithm*}
\endgroup

We could also just manually use the product sum rule as mentioned above.
Also note that it's assuming we're using Euler's totient function, so it won't 
work if Carmichael's totient function was used.
Alternatively, since we know that $\phi (n)$ can be computed using $(p-1)(q-1)$ where $p$ and $q$ are prime numbers, and $p \cdot q = n$, and we have $n$, and we know that
there is only one way to factorise $n$\footnote{As I mentioned earlier, through way
of the Fundamental Thoerem of Arithmetic}, we can instead opt to find $p$ and $q$, so that
we can compute $\phi (n)$ easily using the short method of $(p-1) \cdot (q-1)$

\begingroup
\removelatexerror% Nullify \@latex@error
\begin{algorithm*}[H]
\SetAlgoLined
\KwIn{modulus $n$}
\KwOut{The prime factors of $n$}

\For{numbers in range 2 through ${n}$ inclusive}{
    \If{current number is divisible by $n$}{
        return current number and $\frac{n}{current number}$\;
    }
}
\caption{Prime Factorisation Brute Force}
\end{algorithm*}
\endgroup

Since this method retrieves both $p$ and $q$, it could work with either totient function.

There are a number of optimisations that can be done. For example, 
we only need to loop through until $\left \lfloor{\sqrt{n}}\right \rfloor$ (If you're
not familiar with the notation of floor, it is simply rounding down.)
\footnote{The proof isn't long, try it yourself
by using a proof by contradiction. Also
perhaps starting from $\sqrt{n}$ and then working backwards might be faster.}.
And since we know
that it can be computed using $(p-1)(q-1)$ where $p$ and $q$ are prime numbers, 
we can get a list of prime numbers, and try to see if modulus $n$ divides a certain prime
\footnote{If you recall, you can check if a number divides another by checking
if the remainder is 0 during division} minus 1,
for example $r-1$ where $r$ is a prime number. The trade-off of course being
that you would have to maintain a list of prime numbers.


As a reminder, once $p$ and $q$ are obtained, one could go through the key generation step
as in subsection \hyperlink{subsection.2.3}{\textbf{2.3}}, and then use the Extended
Euclidean Algorithm to obtain $d$.

\section{Further Reading}
At the time of writing, the wikipedia article on RSA\cite{wiki:RSA_(cryptosystem)} is fairly
comprehensive if you want a deeper understanding of RSA. For further information
on attacking RSA, there is this paper {\em ``Twenty Years of Attacks on the RSA
Cryptosystem''} \cite{twenty_years_rsa} which covers methods such as 
``Coppersmith's Attack'' and ``Wiener's Attack''. There is also {\em Revisiting Fermat's Factorization for the RSA Modulus } \cite{gupta2009revisiting} which covers Fermat's
Factorisation for finding $p$ and $q$.
\nocite{*}
\bibliographystyle{plain}
\bibliography{paper}

\end{document}
al
