\documentclass[a4paper,10pt, twocolumn]{article}
\usepackage[utf8]{inputenc}
\usepackage{hyperref}
\usepackage{amsmath}
\usepackage{apacite}

%opening
\title{March Challenge}
\author{Naavin Ravinthran}
\date{}

\begin{document}

\maketitle

\begin{abstract}
This document aims to assist you in completing the March Challenge.
\end{abstract}

\section{Introduction}

\section{The Idea}
MAC Addresses are unique.\footnote{
I may be slightly misleading here, on some devices,
{\emph MAC Randomisation} is a thing, where the MAC Addresses are
randomised when it probes a new network. Android 10 and above does this,
\url{https://source.android.com/devices/tech/connect/wifi-mac-randomization}
but Windows 10 does not \url{https://support.microsoft.com/en-us/windows/how-to-use-random-hardware-addresses-ac58de34-35fc-31ff-c650-823fc48eb1bc}.
} It is not exactly kept secret either, when probing new networks,
the address is given out. If we have a large data collection of
different places and time, we can target a specific device using its
MAC Address and build a profile on it, for example, noting patterns
in where the device is at different times of the day or week. Devices
won't continuously send this information out, but for the purposes
of the challenge let's assume it does.

\section{PCAP File Format}
This is a binary file format. You can open it up in Wireshark and 
search manually, but it would be easier to quickly write a script to
look through it.

There is a global header found at the very beginning of the file 
with the following contents.

\begin{figure}[htbp]
    \centering
    \begin{tabular} {| c | c |}
    \hline
    Name & Type \\
    \hline
    magic\_number & uint32 \\
    \hline
    version\_major & uint16 \\
    \hline
    version\_minor & uint16 \\
    \hline
    thiszone & int32 \\
    \hline
    sigfigs & uint32 \\
    \hline
    snaplen & uint32 \\
    \hline
    network & uint32 \\
    \hline
    \end{tabular}
\caption{Source: }
\url{https://gitlab.com/wireshark/wireshark/-/wikis/Development/LibpcapFileFormat#record-packet-header}


\end{figure}

The magic number is always set to $0xa1b2c3d4$, the major number
is $2$, the minor number is $4$ (as of time of writing), thiszone
indicates the timezone (In practice this is always 0 for GMT/UTC),
sigfigs indicates the significant figures for the time stamp, which
is usually 0, snapen is the ``snapshot length'', and network indicates
the link-layer header type, which will tell you what type it is
and how to interpret the data, eg: Ethernet, Bluetooth, etc. A full list is found here:
\url{http://www.tcpdump.org/linktypes.html}.


And for each packet there is a packet (Record) header:-
\begin{figure}[htbp]
    \centering
    \begin{tabular} {| c | c |}
    \hline
    Name & Type \\
    \hline
    ts\_sec & uint32 \\
    \hline
    ts\_usec & uint32 \\
    \hline
    incl\_len & uint32 \\
    \hline
    orig\_len & uint32 \\
    \hline
    \end{tabular}
\caption{Source: }
\url{https://gitlab.com/wireshark/wireshark/-/wikis/Development/LibpcapFileFormat#record-packet-header}
\end{figure}

ts\_sec is particularly useful to us, because it gives us the unix epoch
time of the packet arrival, which gives us the date and time it was
captured. One would simply need to convert it to a more human-readable
format.

ts\_usec is the microseconds when this packet was captured (as
an offset to the second).

incl\_len is the number of bytes in the packet data in the file.
Don't worry about orig\_len for now.

Once you have that, the data after that would depend on the {\emph 
link-layer header} (See above.). Usually this means that you'll
have to comb through PDFs of Core specifications of different protocols.
But as a tip, you can open in an existing packet capturing software
like Wireshark, and open up the pcap in a hex editor and compare. 
\footnote{As for hex editors, For linux-based distros, I like Okteta.
For Windows, HxD is good}. A hex editor simply opens a binary file
and shows in hex representation what each byte is. Using this method,
you can find where a MAC Address \footnote{In Bluetooth this is
called BD\_ADDR} is located in the packet data, and use that to parse it.

\section{Analysis}
Basically, what you want to do is find any abnormalities.
One way of doing this would be by grouping the MAC Addresses together
(across all the areas),
and seeing the time they are at a particular area. \cite{Unix_time}

Perhaps you can construct a table like this:-
\begin{figure}[htbp]
    \centering
    \begin{tabular} {| c | c |}
    \hline
    Area & Date/Time \\
    \hline
    1 & 2020-01-01 11:15am\\
    \hline
    2 & 2020-01-01 5:00pm\\
    \hline
    1 & 2020-01-02 11:16am \\
    \hline
    3 & 2020-01-02 2:18pm \\
    \hline
    4 & 2020-01-02 2:19pm \\
    \hline
    \ldots & \ldots \\
    \hline
    \end{tabular}
    \caption{Behaviour of a specific Mac Address X}
\end{figure}

Across the different files see if there are any points that 
don't match up. For example, if somebody was in Area 3 at 2:18pm,
and Area 4 in 2:19pm, and it is known that it would be unfeasible
for a regular person to go from Area 3 to Area 4 in a minute as
they are too far away, we would know that would be the speedster.
\footnote{I suppose it is possible that there are multiple devices
spoofing the MAC of other devices, but why would anyone do that?}

After finding the abnormality, the MAC Address doesn't tell us their
identity. We can however, use it to detect patterns in their behaviour,
where they go at what times. Once we have that, it should be easy
to just go to the area and observe who that person is (e.g: 
If you see the MAC Address also regularly visits Area Z on
Thursdays at 3pm, go there at that time on multiple weeks
and find the person who is always there).
In the figure above, perhaps you can surmise that the person
goes to Area 1 every morning at roughly 11:15am.


\subsection{Unix Epoch Time}
A date was chosen, 00:00:00 UTC on 1 Jan 1970, and it was decided
that a format would be created where a 32-bit integer would represent
the number of seconds {\emph after} that time, and this integer
would be used for storing the date and time (It's a signed integer,
so yes it includes negative numbers and dates/times prior to 1970
works). As a side-note, this would unfortunately only work until 
Tuesday 2038-01-19, because after that, the 32-bit integer would
overflow. Some protocols have already migrated to other standards
of representing points of time, but the pcap file hasn't \footnote{
Technically more modern version variants of pcap like pcap-ng has.}

Because of difficult-to-account-for factors like leap-years, change
in timezones, daylight savings time, etc. I'd recommend either using
a library for conversion to a human readable format, or simply using
maths to get the time duration between different time points, bypassing
the need to get the date at all. (e.g: instead of finding 25 Feb 2019 12:00am
and 25 Feb 2019 12:05am and finding the difference, simply subtract
the later date with the earlier date to get $300$ seconds, which
you can easily convert to $6$ minutes)

\nocite{*}
\bibliographystyle{apacite}
\bibliography{bibliography.bib}
\end{document}
